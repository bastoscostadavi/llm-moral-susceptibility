\documentclass{article}

% Allow builds from the repo root by searching both the current dir and articles/
\makeatletter
\def\input@path{{}{articles/}}
\makeatother

\usepackage{mlsys2025}
\usepackage{natbib}
\usepackage{url}
\usepackage{graphicx}
\graphicspath{{../results/}{results/}}
\usepackage{booktabs}
\usepackage{amsmath}
\usepackage{placeins}
\usepackage{listings}
\usepackage{xurl} % better URL line-breaking
\usepackage[hidelinks]{hyperref} % load last

% Listings setup for prompt blocks (two-column friendly, no boxes)
\lstset{
  basicstyle=\ttfamily\scriptsize,
  breaklines=true,
  breakatwhitespace=false, % allow breaking long tokens
  columns=fullflexible,
  keepspaces=true,
  showstringspaces=false,
  tabsize=2,
  frame=none,
  xleftmargin=0pt,
  xrightmargin=0pt,
  aboveskip=0.5em,
  belowskip=0.5em,
}


\title{Moral Robustness and Susceptibility\\ in Large Language Models}




% For double‑blind submission, leave author info empty under the MLSys style.
\author{Davi Bastos Costa, Felippe Alves \& Renato Vicente \\
TELUS Digital Research Hub\\ 
Center for Artificial Intelligence and Machine Learning\\
Institute of Mathematics, Statistics and Computer Science\\
University of São Paulo \\
\texttt{\{davi.costa,felippe.pereira,rvicente\}@usp.br} \\
}

\begin{document}

\maketitle

\begin{abstract}
We study how persona conditioning influences the moral judgments produced by large language models (LLMs). Using the 30-item Moral Foundations Questionnaire (MFQ), we elicit repeated ratings across diverse personas and models, and introduce a benchmark that quantifies two properties: (i) moral robustness (the stability of ratings for personas under repeated sampling), and (ii) moral susceptibility (the sensitivity of MFQ scores under different personas). We find that model family explains most of the variance in moral robustness, and while larger models tend to be more robust within a family, this size effect is modest compared to family-level differences. Susceptibility is more idiosyncratic: it shows weak within-family correlation, varies across moral foundations, and exhibits no consistent size trend. Additionally, we display moral foundation profiles for models in a self (no-persona) condition and report moral foundation profiles for persona characterizations averaged across models, providing a complementary view of the moral effect of personas on model outputs. We release our prompts, runners, and analysis to facilitate replication and comparative evaluation.
\end{abstract}

\section{Introduction}
Reliable benchmarks for the social capabilities of large language models (LLMs) are crucial as models move into interactive, multi-agent settings where outcomes hinge on social intelligence. Recent evaluations probe theory-of-mind, negotiation under asymmetric information, cooperation, and deception through controlled role-play and game-theoretic tasks, e.g.: SOTOPIA for open-ended social interaction \cite{zhou2024sotopia}, MACHIAVELLI for reward–ethics trade-offs \cite{pan2023machiavelli}, NegotiationArena for bargaining \cite{bianchi2024negotiationarena}, ToMBench for structured ToM assessment \cite{chen-etal-2024-tombench}, and Mini-Mafia for emergent deception and detection \cite{costa2025deceivedetectdiscloselarge}. Complementary datasets benchmark social commonsense and moral judgment at scale \cite{sap-etal-2019-social,hendrycks2021ethics}. Motivated by this landscape, we focus on moral judgment as a core facet of social decision-making and alignment.

This paper introduces a benchmark based on the Moral Foundations Questionnaire \citep{moralfoundations2017questionnaires}, a widely used instrument in moral psychology that measures five moral foundations: Harm/Care, Fairness/Reciprocity, In-group/Loyalty, Authority/Respect, and Purity/Sanctity \citep{graham2009liberals,haidt2007when,moralfoundations2017questionnaires}. We formalize two complementary quantities: moral robustness (trial-level rating stability under persona conditioning) and moral susceptibility (between-persona sensitivity of MFQ subscales), both with foundation-level decompositions and uncertainty estimates. We also provide a simple, reproducible evaluation protocol: a role-playing runner that elicits repeated MFQ ratings under diverse personas, together with released prompts, scripts, and analysis to enable replication. Applying this framework across contemporary model families and sizes, we find that family identity explains most of the variance in robustness; within families, larger variants tend to be only modestly more robust. Susceptibility is more idiosyncratic: it shows weak within-family correlation, varies across foundations, and exhibits no consistent size trend. In our runs, Claude Sonnet is the most robust across foundations, Grok models are among the least robust, and Grok-4-fast shows the highest susceptibility overall.

Recent MFQ-based studies profile LLM value orientations and alignment. \citet{abdulhai-etal-2024-moral} adapt MFQ prompts to derive foundation scores, compare them to human surveys, and show that targeted prompts can shift profiles and affect downstream donations. \citet{nunes2024hypocrites} combine MFQ with MFV to reveal inconsistencies between abstract and concrete judgments. \citet{aksoy2024whose} use MFQ-2 across eight languages to expose cultural/linguistic variability, and \citet{bajpai2024insights} compare MFQ-20 and moral competence between humans and chatbots, finding LLMs emphasize individualist foundations and lag human competence. In parallel, MoralBench \citep{ji2025moralbenchmoralevaluationllms} offers a broad task suite; our MFQ persona framework complements it by isolating persona-driven shifts relative to a self baseline. For applied deployments, it remains useful to understand the baseline moral profile of the models being used; accordingly, we also report model-level MFQ profiles (self/no-persona), complementing broad suites such as MoralBench and extending MFQ profiling to more advanced, state-of-the-art models. In addition, we provide MFQ profiles for different personas averaged across models to surface typical persona-driven shifts. For comparability, we further present z-score–normalized summaries across models.

\section{Moral Robustness and Susceptibility Benchmark}

We define a benchmark to evaluate the moral robustness and moral susceptibility of LLMs. Moral robustness, is the stability of MFQ ratings across personas under repeated sampling, precisely defined in \eqref{eq:robustness}. Moral susceptibility is the sensitivity of MFQ scores under different personas, defined in \eqref{eq:overall-susceptibility}.

\subsection{Moral Foundation Questionnaire}
The Moral Foundation Questionnare \citep{moralfoundations2017questionnaires} comprises 30 items split into two sections: 15 relevance judgments (how relevant specific considerations are when deciding right from wrong) and 15 agreement statements (level of agreement with moral propositions) \citep{graham2011mfq,moralfoundations2017questionnaires}. Items map to five moral foundations (Harm/Care, Fairness/Reciprocity, In-group/Loyalty, Authority/Respect, Purity/Sanctity). Subscale scores are computed by averaging the items associated with each foundation within each section and then combining sections (mean of relevance and agreement for that foundation).

In our implementation, each prompt instructs the model to produce a leading integer in \([0,5]\) reflecting either relevance (0=not at all, 5=extremely) or agreement (0=strongly disagree, 5=strongly agree), followed by free-text reasoning. Ratings are parsed by extracting the first digit \([0,5]\) from the response. Figure~\ref{fig:mfq-profiles} illustrates the resulting MFQ relevance profile across models using the self (no-persona) baseline, specifically, the models where prompted exclusively with the MFQ questions.

\begin{figure}[t]
  \centering
  \includegraphics[width=\linewidth]{../results/moral_foundations_relevance_profiles.png}
  \caption{Average Moral Foundation Profile Across Models (self/no-persona baseline). Points show mean relevance per foundation; error bars denote standard errors across items within each foundation.}
  \label{fig:mfq-profiles}
\end{figure}

\begin{figure}[t]
  \centering
  \includegraphics[width=\linewidth]{../results/persona_moral_foundations_relevance_profiles.png}
  \caption{Moral foundation relevance profiles for five randomly selected personas, averaged across models. This visualization highlights an averaged effect of persona identity (persona\_id) on MFQ relevance patterns. See the Personas appendix for descriptions; indices match the zero-based persona\_id used in our runs.}
  \label{fig:persona-mfq-profiles}
\end{figure}

Figure~\ref{fig:persona-mfq-profiles} illustrates the resulting MFQ relevance profile average over all models for different personas. It gives an average characterization of the moral persona conditioning on models. The full per-persona and per-model MFQ ratings are available in CSV files in our GitHub repository \cite{costa2025llmms}. 

\subsection{Experimental Methodology}
We use a simple, reproducible runner that iterates through MFQ items for a list of personas and repeats each item multiple times to characterize response variability. Concretely:
\begin{itemize}
  \item \textbf{Personas:} A JSON file provides persona descriptions in plain text, extracted from \citep{ge2025scalingsyntheticdatacreation}. By default, each persona is used as-is and identified by its index. See Appendix~\ref{app:personas} for the evaluated personas.
  \item \textbf{Prompting:} For each persona and item, the model receives a roleplaying instruction plus the MFQ question. Exact prompt templates are provided in Appendix~\ref{app:prompts}.\footnote{We query one MFQ item at a time rather than the full questionnaire in a single prompt to avoid sequence- and order-dependent effects. Studying how MFQ responses change when posed as a single questionnaire and under randomized item orders is interesting in its own right and left for future work.}
  \item \textbf{Repetitions:} Each persona--question pair is queried \(n\) times (default \(n=10\)) to estimate within-persona variability and uncertainty in the ratings.
  \item \textbf{Decoding:}  The prompt requests a leading integer rating in \([0,5]\) and we set \texttt{max\_tokens} to 1 to elicit short, just rating outputs. Ratings are parsed with a conservative regex with failures recorded as \(-1\) (see Section~\ref{sec:failures} for details).
  \item \textbf{Logging:} Each response is streamed to CSV with fields: persona\_id, question\_id, run\_index, rating, and timestamp.
  \item \textbf{Models:} We included: Claude Haiku 4.5, Claude Sonnet 4.5, Gemini 2.5 Flahs Lite, GPT-4.1, GPT-4.1 Mini, GPT-4.1 Nano, GPT-4o, GPT-4o Mini, Grok-4 and Grok-4 Fast.
\end{itemize}


\subsection{Statistical Analysis}



This section formalizes the quantities we compute from the MFQ runs and how we summarize them into moral robustness and susceptibility metrics.

Let \(\mathcal{P}\) be the set of personas, \(\mathcal{Q}\) the set of 30 scored MFQ items, and \(R\) the number of repeated queries per persona--item pair. For persona \(p\), item \(q\), and repetition \(i=1,\ldots,R\), let \(y_{pqi}\in\{0,\ldots,5\}\) be the parsed rating.

For each persona--item pair we compute the sample mean and the standard deviation across repetitions
\begin{align}
  \bar{y}_{pq} &= \frac{1}{R} \sum_{i=1}^{R} y_{pqi}, \label{eq:persona-question-mean}\\
  u_{pq} &= \sqrt{\frac{1}{R-1} \sum_{i=1}^{R} \big(y_{pqi} - \bar{y}_{pq}\big)^2}, \label{eq:persona-question-se}
\end{align}

\paragraph{Moral robustness} We summarize within-pair variability by averaging the SDs in \eqref{eq:persona-question-se} over personas and items
\begin{equation}
  \bar{u} = \frac{1}{|\mathcal{P}|\,|\mathcal{Q}|} \sum_{p \in \mathcal{P}} \sum_{q \in \mathcal{Q}} u_{pq}.\label{eq:mean-uncertainty}
\end{equation}
Our robustness index is the reciprocal
\begin{equation}
  R = \frac{1}{\bar{u}}.\label{eq:robustness}
\end{equation}
Let the (sample) standard deviation of the \(u_{pq}\) values be
\begin{equation}
  s_u = \sqrt{\frac{1}{|\mathcal{P}|\,|\mathcal{Q}| - 1} \sum_{p \in \mathcal{P}} \sum_{q \in \mathcal{Q}} \Big(u_{pq} - \bar{u}\Big)^2}.\label{eq:uncertainty-sd}
\end{equation}
Then the SE of \(\bar{u}\) is \(\sigma_{\bar{u}} = s_u / \sqrt{|\mathcal{P}|\,|\mathcal{Q}|}\) which we propagate to get an estimate for robustness SE:
\begin{equation}
  \sigma_R = \frac{\sigma_{\bar{u}}}{\bar{u}^2}.
  \label{eq:robustness-se}
\end{equation}
Foundation-level robustness repeats \eqref{eq:mean-uncertainty}--\eqref{eq:robustness-se} with sums over \(\mathcal{Q}_f\).


\paragraph{Moral susceptibility} To stabilize estimates across many personas, we partition \(\mathcal{P}\) into \(G\) disjoint groups \(\mathcal{P}_1,\ldots,\mathcal{P}_G\) of equal size. For each item \(q\) and group \(g\), we compute the sample standard deviation of persona means
\begin{align}
  & s_{qg} = \sqrt{\frac{1}{|\mathcal{P}_g|-1} \sum_{p \in \mathcal{P}_g} \Big(\bar{y}_{pq} - \bar{y}_{gq}\Big)^2},
  \label{eq:question-dispersion}
\end{align}
with $\bar{y}_{gq}$ the average over $\mathcal{P}_g$, i.e.: 
\begin{align}
  \bar{y}_{gq} = \frac{1}{|\mathcal{P}_g|} \sum_{p \in \mathcal{P}_g} \bar{y}_{pq}.
\end{align}
From $s_{qg}$ we obtain a group-level susceptibility sample
\begin{equation}
  S_g = \frac{1}{|\mathcal{Q}|} \sum_{q \in \mathcal{Q}} s_{qg}.\label{eq:group-susceptibility}
\end{equation}
The reported susceptibility is the mean over groups
\begin{equation}
  S = \frac{1}{G} \sum_{g=1}^{G} S_g,\label{eq:overall-susceptibility}
\end{equation}
with its standard error estimated from the between-group variability
\begin{equation}
  \sigma_S = \frac{\sqrt{\frac{1}{G-1} \sum_{g=1}^{G} (S_g - S)^2}}{\sqrt{G}}.\label{eq:susceptibility-se}
\end{equation}
Foundation-specific susceptibilities reuse \eqref{eq:question-dispersion}--\eqref{eq:susceptibility-se} after restricting \(\mathcal{Q}\) to the item subset \(\mathcal{Q}_f\) for foundation \(f\).

\paragraph{Cross-model normalization} To facilitate comparison, we also present the z-scores that summarize relative performance across models. The $z$-score for metric $M\in \{S,R\}$ is
\begin{equation}
  z_{M} = \frac{M - \mu_M}{\sigma_M},
  \label{eq:zscore}
\end{equation}
where $M$ is the models's score, $\mu_M$ is the mean, and $\sigma_M$ is the standard deviation over different models. The uncertainty of $z_M$ is propagated from that of $M$, $\mu_M$ and $\sigma_M$.

\subsection{Failures to Respond}
\label{sec:failures}
We treat rows flagged as failed generations as unusable signal: any trial with a positive failure flag (e.g., \texttt{failures}~$>$~0) is discarded. Whenever a repetition produced an invalid response, we immediately reran the prompt, allowing up to three attempts per repetition. Most failures arose when models did not follow the instruction and appended the rating after their reasoning; increasing the completion budget (\texttt{max\_tokens}) typically recovered a valid rating, often in a single additional attempt. In a few cases, models refused to provide a rating for a given persona--question pair. Across all runs, nine personas were affected; we excluded these personas from the analysis aggregates.

In practice, the following personas met the complete-failure criterion and were removed from the analysis set: \texttt{\{29, 42, 44, 51, 66, 75, 86, 90, 95\}}. We then choose the following grouping (\(91 = 7 \times 13\)) for estimating the moral susceptibility and its uncertainty.

Table~\ref{tab:failures_by_model} reports, for completeness, the total number of failed attempts (summing the \texttt{failures} column) per dataset; we list only datasets with non-zero totals.

\begin{table}[t]
  \centering
  \caption{Total failure counts per dataset (raw reruns under \texttt{data/}).}
  \label{tab:failures_by_model}
  \begin{tabular}{lr}
    \toprule
    Dataset & Total failures \\
    \midrule
    claude-sonnet-4-5      & 37  \\
    gemini-2.5-flash-lite  & 344 \\
    gpt-4.1                & 4   \\
    gpt-4o                 & 37  \\
    gpt-4o-mini            & 202 \\
    \bottomrule
  \end{tabular}
\end{table}

Rather than estimating item ratings via repeated sampling (10 trials), a more principled alternative is to use the model’s next-token distribution to directly compute an expected rating. Given the question prompt (that includes a the instruction that the response should begin with the rating from 0--5), let \(p_n = p(n\mid\text{prompt})\) denote the probability that the next token is the digit \(n\). Then estimate
\begin{equation}
  \hat r = \frac{p_1+2p_2+3p_3+4p_4+5p_5}{p_0+p_1+p_2+p_3+p_4+p_5}
\end{equation}
In expectation, this equals the average that our 10-trial procedure approximates, while avoiding failures and sampling variance. Implementing this requires access to token-level probabilities/log-probabilities from provider APIs. Care is needed around tokenization (e.g., space-prefixed digits or multiple token aliases) and to ensure probabilities are measured at the very first output position.



\section{Results}

We present moral robustness and susceptibility by model both overall and by foundation, plus a $z$-score summary table across models.

% Merged summary + z-score table across models
\begin{table*}[t]
  \centering
  \caption{Overall susceptibility and robustness by model with z-scores (mean $\pm$ SE; $Z$ computed across models).}
  \label{tab:summary_by_model_with_z}
  \begin{tabular}{lcccc}
    \toprule
    Model & Robustness ($\pm$) & Robustness $Z$ ($\pm$) & Susceptibility ($\pm$) & Susceptibility $Z$ ($\pm$) \\
    \midrule
    claude-haiku-4-5 & $32\pm 4$ & $0.2\pm 0.1$ & $0.78\pm 0.06$ & $0.5\pm 0.9$ \\
    claude-sonnet-4-5 & $108\pm 10$ & $2.7\pm 0.4$ & $0.71\pm 0.05$ & $-0.6\pm 0.7$ \\
    gemini-2.5-flash-lite & $26\pm 2$ & $0.04\pm 0.06$ & $0.81\pm 0.04$ & $0.9\pm 0.6$ \\
    gpt-4.1-mini & $11.3\pm 0.4$ & $-0.44\pm 0.01$ & $0.78\pm 0.04$ & $0.4\pm 0.7$ \\
    gpt-4.1-nano & $12.3\pm 0.6$ & $-0.41\pm 0.02$ & $0.69\pm 0.05$ & $-1.0\pm 0.8$ \\
    gpt-4.1 & $14.3\pm 0.6$ & $-0.34\pm 0.02$ & $0.77\pm 0.05$ & $0.3\pm 0.7$ \\
    gpt-4o-mini & $12.9\pm 0.6$ & $-0.39\pm 0.02$ & $0.62\pm 0.04$ & $-2.0\pm 0.6$ \\
    grok-4-fast & $3.33\pm 0.06$ & $-0.699\pm 0.002$ & $0.86\pm 0.05$ & $1.6\pm 0.8$ \\
    grok-4 & $3.31\pm 0.05$ & $-0.700\pm 0.002$ & $0.75\pm 0.03$ & $-0.1\pm 0.5$ \\
    \bottomrule
  \end{tabular}
\end{table*}


\subsection{Moral Robustness}
We quantify trial-level stability by first computing the sample standard deviation across repetitions for each persona–question pair (Eq.~\ref{eq:persona-question-se}), averaging these to obtain \(\bar{u}\) (Eq.~\ref{eq:mean-uncertainty}), and defining robustness as \(R = 1/\bar{u}\) (Eq.~\ref{eq:robustness}) with uncertainty propagated via Eq.~\ref{eq:robustness-se}.

% Robustness (overall + five foundations)
\begin{figure*}[!t]
  \centering
  \includegraphics[width=0.3\linewidth]{../results/robustness_overall.png}\hfill
  \includegraphics[width=0.3\linewidth]{../results/robustness_harm_care.png}\hfill
  \includegraphics[width=0.3\linewidth]{../results/robustness_fairness_reciprocity.png}\\[0.75em]
  \includegraphics[width=0.3\linewidth]{../results/robustness_in_group_loyalty.png}\hfill
  \includegraphics[width=0.3\linewidth]{../results/robustness_authority_respect.png}\hfill
  \includegraphics[width=0.3\linewidth]{../results/robustness_purity_sanctity.png}
  \caption{Six-panel summary of robustness (inverse of average per-item standard deviation across repetitions). Top row: overall benchmark, Harm/Care, and Fairness/Reciprocity. Bottom row: In-group/Loyalty, Authority/Respect, and Purity/Sanctity. Error bars show propagated SE via delta method; higher values indicate greater rating stability.}
  \label{fig:robustness}
\end{figure*}

Robustness exhibits clear within-family structure across models. We observe a strong correlation by model family (e.g., families cluster together across foundations), with the Claude family consistently the most robust; notably, \emph{Claude Sonnet} outperforms all others by a sizeable margin across foundations. In contrast, the Grok models are the least robust on average. We do, however, observe a modest size effect: within a family, larger variants tend to be more robust (e.g., GPT-4.1 > mini > nano; Grok-4 > Grok-4-fast), but these differences are small relative to the family-level gaps. These trends are visible in Figure~\ref{fig:robustness} and summarized in the z-score table (Table~\ref{tab:summary_by_model_with_z}).

\subsection{Moral Susceptibility}
We assess between-persona sensitivity by computing within-group dispersion of persona means per item (Eq.~\ref{eq:question-dispersion}), averaging across items to form group-level samples (Eq.~\ref{eq:group-susceptibility}), and reporting the across-group mean and its SE (Eqs.~\ref{eq:overall-susceptibility}--\ref{eq:susceptibility-se}).

% Susceptibility (overall + five foundations)
\begin{figure*}[!t]
  \centering
  \includegraphics[width=0.3\linewidth]{../results/susceptibility_overall.png}\hfill
  \includegraphics[width=0.3\linewidth]{../results/susceptibility_harm_care.png}\hfill
  \includegraphics[width=0.3\linewidth]{../results/susceptibility_fairness_reciprocity.png}\\[0.75em]
  \includegraphics[width=0.3\linewidth]{../results/susceptibility_in_group_loyalty.png}\hfill
  \includegraphics[width=0.3\linewidth]{../results/susceptibility_authority_respect.png}\hfill
  \includegraphics[width=0.3\linewidth]{../results/susceptibility_purity_sanctity.png}
  \caption{Six-panel summary of moral susceptibility (mean $\pm$ SE across persona groups). Top row: overall benchmark, Harm/Care, and Fairness/Reciprocity. Bottom row: In-group/Loyalty, Authority/Respect, and Purity/Sanctity. Higher values indicate larger persona-driven shifts in MFQ subscale scores.}
  \label{fig:susceptibility}
\end{figure*}

Susceptibility is more idiosyncratic: we do not observe strong correlation within model families, and rankings vary across foundations. The most susceptible model overall is \emph{Grok-4-fast}, indicating larger persona-driven shifts relative to peers. Across GPT-4.1 variants there is no consistent size pattern (normal, mini, nano are comparable). See Figure~\ref{fig:susceptibility} for the multi-foundation view and Table~\ref{tab:summary_by_model_with_z} for the corresponding z-scores. Complete moral foundation profiles for each persona and model are available in our GitHub repository \citep{costa2025llmms}.


\section{Conclusion}
We propose a principled benchmark for quantifying persona-driven shifts in LLM moral judgments using the MFQ. Our framework separates susceptibility (persona sensitivity) and robustness (rating stability), supports multiple model classes, and relies on transparent, easily repeatable procedures. Future work includes expanding persona taxonomies, stress-testing prompt formats, modeling reasoning content jointly with ratings, and correlating susceptibility with downstream alignment and safety outcomes.



% Use the official MLSys 2025 bibliography style when available
\bibliographystyle{mlsys2025}
\bibliography{references}


\appendix

\section{Prompt Format}
\label{app:prompts}
We use a simple roleplaying wrapper with MFQ item prompts. The item texts are sourced from the Moral Foundations Questionnaire \cite{moralfoundations2017questionnaires} and are not repeated here. Below we show the exact prompt formats used.

\subsection*{Persona wrapper}

\begin{lstlisting}
You are roleplaying as the following persona:
"<persona description>"

Please answer the following question.
\end{lstlisting}

\subsection*{Relevance prompt template}

\begin{lstlisting}
When deciding whether something is right or wrong, how relevant is the
following consideration to your thinking: '<CONSIDERATION>'. Please rate the
consideration using this scale:
 0 = not at all relevant,
 1 = not very relevant,
  2 = slightly relevant,
 3 = somewhat relevant,
 4 = very relevant,
 5 = extremely relevant.

Your response should start with an integer from 0 to 5, followed by your
reasoning.
\end{lstlisting}

\subsection*{Agreement prompt template}
\begin{lstlisting}
Please indicate your level of agreement with the following statement:
'<STATEMENT>'. Please rate the statement using this scale:
 0 = strongly disagree,
 1 = moderately disagree,
 2 = slightly disagree,
 3 = slightly agree,
 4 = moderately agree,
 5 = strongly agree.

Your response should start with an integer from 0 to 5, followed by your
reasoning.
\end{lstlisting}

\section{Personas}
\label{app:personas}
We evaluated models under a diverse set of personas to probe persona-driven shifts in MFQ responses. We include a numbered sample below; indices match the zero-based persona identifiers (\texttt{persona\_id}) used in our runs. The complete list is provided with the artifact (\texttt{personas.json}). Personas were sampled from prior work on large-scale persona generation \citep{ge2025scalingsyntheticdatacreation}.

\IfFileExists{appendix_personas.tex}{\input{appendix_personas}}{\small\emph{Personas list will appear after running the generator script.}}

\end{document}
